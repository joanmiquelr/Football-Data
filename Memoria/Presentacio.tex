% !TEX root=MemoriaTFG.tex

\chapter{La presentació del Treball Final de Grau}\label{presentació}

\section{Principis bàsics}
\emph{Què és el que fa que una presentació de \ac{TFG} sigui bona?} L'objectiu de qualsevol presentació és donar a conèixer un determinat missatge fent que l'audiència l'entengui i el recordi \cite{Blair91,Padgett08}. En concret, la presentació del \ac{TFG} ha de permetre a l'alumne sintetitzar el treball que ha estat realitzant durant varis mesos i donar-lo a conèixer als membres del tribunal per tal que aquests puguin determinar fins a quin punt s'han assolit els objectius plantejats.

Les qualitats més rellevants que ha de tenir una presentació tècnica/científica per ser bona són (\cite{Blair91,Padgett08,Arcy88,Rice04}):
\begin{itemize}\tightlist
  \item concisa i precisa: cal identificar les idees clau (objectius i assoliments del treball realitzat) i comunicar-les amb precisió estilística i tècnica.
  \item organitzada i estructurada: una organització acurada i coherent del material a presentar facilita enormement la seva comprensió per part de l'audiència.
  \item adequada a l'audiència i capaç de captar i mantenir la seva atenció: s'ha d'\emph{en\-gan\-xar} l'audiència per tal que entengui millor el que es vol expressar. Per això el missatge s'ha de caracteritzar per la seva:
      \begin{enumerate}\tightlist
        \item claredat
        \item simplicitat
        \item correcció
        \item precisió
      \end{enumerate}
  \item exhaustivament preparada: quan les idees es comuniquen d'una manera pobra no s'obté cap benefici del nostre esforç, ni per part nostra ni per part de l'audiència.
\end{itemize}

L'objectiu serà despertar prou interès en el tribunal com per aconseguir captar veritablement la seva atenció i donar-li a conèixer el problema a resoldre, la solució adoptada i els resultats obtinguts.

\section{Bones pràctiques}
\emph{Quines són les millors passes a fer per preparar una bona presentació de \ac{TFG}?}

\subsection{Les transparències}

        A l'hora de preparar la presentació del \ac{TFG} podem caure en la temptació d'explicar el contingut complet del nostre treball, però l'objectiu ha de ser deixar clars tan sols els punts més importants. Si aquests no són massa nombrosos, serà possible explicar-los de manera clara, mentre que si pretenem mostrar massa idees tan sols aconseguirem confondre l'audiència. Els detalls sobre el treball desenvolupat es troben a la memòria del \ac{TFG} i no s'han d'explicar a la presentació. A l'hora de preparar les explicacions s'ha de tenir present que les nostres idees sempre ens semblen molt més simples a nosaltres que a aquells que encara no les entenen. La clau d'una bona presentació es fonamenta en descriure de manera adequada:
        \begin{itemize}\tightlist
            \item el problema a resoldre: les necessitats del projecte, la seva motivació, explicant la situació i l'entorn en el qual neix, el perquè és necessari
            \item la solució adoptada i els resultats obtinguts
            \item les conclusions del treball i les possibles línies de futur
        \end{itemize}

        És fonamental tenir presents les característiques de l'audiència i proporcionar-li la informació que necessita per tal que pugui entendre la presentació. Sempre és bo començar amb una revisió d'aquells conceptes bàsics que, tot i ser probablement coneguts per l'auditori, facilitaran la comprensió de la presentació. Com a part de la preparació s'han d'avaluar no només els coneixements sinó també els interessos, necessitats i valors de l'audiència (criteris de puntuació del \ac{TFG}).

        Un cop identificats els punts clau que volem deixar clars i les característiques de l'audiència, es realitzarà un esbós del contingut de les transparències. A partir d'aquest esbós es va perfilant l'estructura de la presentació, que ha de seguir un fil argumental lògic i coherent. El contingut de les transparències haurà de marcar en tot moment com es va avançant per aquest fil, evitant que l'audiència es perdi i deixi d'estar atenta. Mantenint sempre aquest propòsit en ment es decideixen els continguts de les transparències, que només inclouran aquelles explicacions necessàries per a poder assolir el nostre objectiu, que no és altre que el de transmetre a l'auditori els punts que hem escollit com a idees clau. Aquells continguts que no siguin imprescindibles per a aconseguir aquesta finalitat no s'inclouran en les transparències.

        Tot i que la normativa estableix una durada màxima de 45 minuts, és recomanable que no excedeixi els 30 o 35 minuts, deixant uns 15 minuts pel torn de preguntes. S'han de redactar les transparències tenint en compte que generalment és recomanable dedicar aproximadament entre dos i tres minuts a cada transparència \cite{Padgett08}. Cal tenir en compte que l'audiència sol perdre l'atenció passats uns 10 o 15 minuts d'activitat similar \cite{Padgett08}. Això significa que el rellotge de l'atenció s'ha de tornar a iniciar en el transcurs de l'exposició oral, i per aconseguir-ho cal donar algun tipus de gir en la presentació, la qual cosa es pot aconseguir, per exemple, amb algun tipus de demostració o de recapitulació.

        En la redacció del text de les transparències es procurarà que tant els títols com les frases siguin directes i curts. Els paràgrafs han de ser breus i l'estil utilitzat ha de ser impersonal i objectiu. A més s'ha de procurar minimitzar el text present a les transparències i s'evitarà sempre copiar paràgrafs complets de la memòria a la presentació. És fonamental assegurar-se de la total correcció ortogràfica i gramatical.

        Les taules, gràfics, diagrames i imatges que mostren el que volem expressar, són eines de comunicació molt valuoses. Cal indicar clarament a l'audiència el que es mostra en aquests elements visuals, mai s'ha de donar per suposat que ja ho sap. D'aquesta manera qui atén les nostres explicacions es beneficia d'escoltar i de veure simultàniament el nostre missatge. Si s'ha de fer alguna demostració de l'aplicació o producte final és convenient gravar-ho en vídeo per tal d'evitar possibles inestabilitats del software o problemes amb els servidors.

        Convé provar les diapositives en el projector i el PC de la sala on es farà la defensa, ja que poden canviar colors, ... També cal confirmar que la mida de lletra és adequada i que els gràfics es visualitzen de manera clara i que són inte\l.ligibles. S'evitaran els colors estridents i els dissenys de diapositiva que restin claredat al contingut de les transparències.

\subsection{L'exposició i la defensa}

        És fonamental preparar-se bé la presentació, explicitant el que s'ha de dir a cada transparència. L'alumne ha de tenir molt clares les explicacions que pretén donar als membres del tribunal, ja que en cas que tingui aspectes confusos mai podrà transmetre'ls amb claredat. És recomanable practicar-la repetides vegades tant tot sol com davant d'altres persones, per tal d'agafar confiança i fluïdesa, així com per ajustar-se al temps disponible. S'evitarà recitar de memòria la presentació, però sí que és bo memoritzar algunes paraules o frases clau per a cada transparència. De totes maneres, per tal de superar el pànic escènic, que és especialment gran a l'inici, sí pot ser convenient aprendre's les dues o tres primeres transparències. És aconsellable fer un assaig general de la defensa en la mateixa sala i amb projector uns quants dies abans de la presentació definitiva, amb l'assistència del director del \ac{TFG}, ja que aquest podrà fer totes aquelles recomanacions i correccions que consideri oportunes i que seran de gran ajut per l'alumne.

        L'estil i el to de l'exposició oral han d'afavorir que l'audiència mantingui la seva atenció. Per això el llenguatge utilitzat ha de ser apropiat no només a la disciplina sinó també a l'audiència. S'utilitzarà un llenguatge tècnicament correcte, evitant utilitzar expressions co\l.loquials, repeticions i falques. S'ha de parlar amb autoritat i confiança, mostrant i demostrant el coneixement i domini del treball presentat. S'ha de tenir esment al to de veu i al ritme amb què es parla, utilitzant pauses, per exemple entre les distintes seccions, que afavoreixin la comprensió de l'exposició. L'orador ha de fer un ús adequat del contacte visual, tant de manera individual com de cap al grup que l'està escoltant. De la mateixa manera la posició corporal i els gestos, així com l'aparença, hauran de ser apropiats. No és convenient moure's excessivament durant la presentació, ni interposar-se entre la pantalla on es projecten les diapositives i els membres del tribunal. Un punter pot ajudar a centrar l'atenció sobre algun punt específic d'una transparència, però no s'ha abusar d'aquest recurs.

        En general, qui més sap del TGF és el propi projectista, a part del seu director, de manera que el torn de preguntes dels membres del tribunal no ha de provocar cap temor en l'alumne. És convenient anar a totes les defenses de \ac{TFG} possibles, i fixar-se en allò que l'alumne fa bé i malament, el tipus de preguntes que fa el tribunal, les respostes, \ldots\ És recomanable dur aigua a la presentació.

\section{Estructura de la presentació del Treball Final de Grau}

L'estructura típica de la presentació constarà, de la mateixa manera que la memòria, de quatre parts fonamentals: introducció, desenvolupament, resultats i conclusions. Tot i així és recomanable completar aquesta estructura bàsica començant amb una portada seguida d'un índex.
\subsection{Portada} Inclourà el títol del \ac{TFG} així com el nom dels autors, dels directors i dels tutors si s'escau, del departament, de la universitat, del títol acadèmic al qual s'opta i de la data de la presentació. Una bona opció és mantenir projectada des de l'inici aquesta primera transparència, mentre els membres del tribunal i l'audiència entren a la sala.
\subsection{Índex} Consistirà en una breu descripció dels principals punts de la presentació. Pretén preparar a l'audiència per tal que vagi identificant fàcilment els punts importants a mesura que es van cobrint al llarg de la presentació. Permet comunicar l'organització de la presentació. És bo que el guió estigui sempre visualment present, per exemple a la capçalera de les transparències.
\subsection{Desenvolupament} Explicarà el treball realitzat (problema a resoldre i solució adoptada), seguint un fil argumental lògic i coherent, i de manera que capti i mantingui l'atenció i l'interès de l'audiència.
\subsection{Resultats} Mostraran la validesa de la solució adoptada, generalment amb l'ajuda de taules i figures. Per assegurar la seva comprensió per part de l'audiència és imprescindible explicitar allò que es presenta en les taules i figures. A més és necessari assegurar-se de la seva correcta visualització amb el projector. No és necessari mostrar tots els resultats presentats a la memòria, sinó tan sols aquells que condueixin a la correcta comprensió del treball realitzat i dels resultats obtinguts.
\subsection{Conclusions} Resumiran la presentació completa. Després d'haver explicat els principals punts en el cos de la presentació, l'audiència pot haver-se perdut en els detalls, així és que cal reiterar la importància del problema que havíem plantejat i els principals aspectes de la solució proposada. Així mateix, també es poden descriure breument les possibles línies de treball futur.
