%% ----------------ATENCIÓ-----------------------
%% Aquest document està preparat per codificació 
%% UTF-8. Si es llegeix amb un editor 
%% preparat per una altra codificació alguns 
%% caràcters no es veuran correctament.
%%---------------------------------------------- 
%%
%% Aquest document vos pot servir de base per la
%% realització de la memòria del treball final de
%% grau. Es recomana seguir els consells que 
%% apareixen en els diferents comentaris del 
%% document. Convindrà mantenir l'estructura
%% d'aquest exemple i introduir canvis sols 
%% allà on s'indiqui.
%%
%%-----------------------------------------------
%%
%% La plantilla preparada per a la realització de
%% la memòria és la classe LaTeX TFGEPSUIB.cls
%% que es fonamenta en la classe memoir. Aquesta
%% emula alguns `packages` que, per tant, no serà
%% necessari incloure dins el vostre document. El
%% manual de memoir explica quins són aquests
%% paquets (per exemple booktabs, array, tabluarx). 
%% Per altra banda, TFGEPSUIB també carrega 
%% automàticament els paquets, fontenc, xcolor, 
%% graphicx i microtype.
%%
%%-----------------------------------------
%% Opcions de la classe TFGEPSUIB:
%%
%% Per defecte, la classe TFGEPSUIB suposa
%% que la memòria es redactarà en català i que 
%% els estudis són els de Grau en Enginyeria
%% telemàtica. 
%%
%% Si preferiu utilitzar el castellà o anglés
%% per a la redacció de la memòria podeu
%% fer-ho indicant-ho a les opcions.
%\documentclass[catalan]{TFGEPSUIB}
\documentclass[spanish,GMAT]{TFGEPSUIB}
%%
%% Per indicar quins són els estudis pels quals
%% es redacta la memòria heu d'incloure una de
%% les següents opcions:
%%
%% GTEL - Grau en Enginyeria Telemàtica
%% GMAT - Grau de Matemàtiques
%% GINF - Grau d'Enginyeria Informàtica
%% GEDI - Grau d'Enginyeria d'Edificació
%% GELE - Grau d'Eng. Elec. Ind. i Automàtica
%% GAGR - Grau en Eng. Agroali. i del Medi Rural
%%
%% Per escriure la memòria en castellà pels estudis
%% de matemàtiques, la línia inicial serà
%\documentclass[spanish,GMAT]{TFGEPSUIB}
%%------------------------------------------

% Per incloure expressions matemàtiques en
% el document és recomanable usar els paquets
%\usepackage{amsmath,mathtools} 

% Per incloure fragments de codi hi ha diferents
% paquets disponibles. Triau el que vos sigui 
% més convenient, per exemple listings. 
%\usepackage{listings} 

% Amb tcolorbox podeu definir caixes per llistats,
% teoremes, exemples, ...
\usepackage{tcolorbox}

% Si voleu que les referències bibliogràfiques
% apareguin amb el format "autor-any" en comptes 
% de "número" heu d'usar el paquet natbib.
%\usepackage[round,colon,sort&compress]{natbib} 

% En general, la documentació tècnica pot
% incloure molts acrònims. Per això es recomana
% usar el següent paquet. Consultau el corresponent
% manual per saber com s'usa.
\usepackage[printonlyused]{acronym}

% Les diferents unitats de mesura tenen un
% format estàndard de representació que convé
% respectar. Per això s'usa el paquet `siunitx`.
% També permet representar nombres en notació
% científica i alinear correctament els valors 
% numèrics a les taules. Pegau una ullada al 
% manual. Si no voleu usar-lo comentau la línia. 
\usepackage{siunitx}

% Com que és convenient que el paquet hyperref
% sigui un dels darrers en carregar-se, si voleu 
% afegir nous paquets, feis-ho a continuació.
%\usepackage{}

% El paquet "hyperref" crea enllaços automàticament 
% dins el document. Aquests enllaços permeten la 
% navegació a través de les diferents referències 
% figures, bibliografia, fórmules, índex, ...
% tant sols assenyalant-los amb el ratolí. 
\usepackage[backref, colorlinks=true, all colors=black]{hyperref}

% Si voleu que els enllaços apareguin en colors diferents,
% eliminau l'opció "all colors=black". 

%----------------------------------------------
% Dades de la Portada amb els valors adequats
% La portada inclou informació del estudis,
% títol del projecte, autor(s) del projecte,
% tutor(s) i data. 
%
% Els estudis ja s'han definit amb la opció
% corresponent, però si el projecte
% és d'uns altres estudis no definits a les
% opcions, com són, per exemple, els dels plans
% anteriors, sempre podeu usar la comanda \estudis
% per definir-los.
% Per exemple, per l'enginyeria tècnica en 
% telemàtica podeu usar:
%\estudis{Enginyeria Tècnica en Telecomunicacions, especialitat Telemàtica}


% Aquí podeu posar el títol de la vostra memòria
\title{Recomanacions per a la realització del Treball Final de Grau a l'Escola Politècnica Superior}

% Nom de l'autor del TFG.
\author{Joan Miquel Rubi Garcias}

% La comanda \tutor mostra el nom del director
% a la portada interior. Si hi ha més d'un tutor
% caldrà fer un petit canvi. Demanau ajuda. 
\tutor{Nelson Alirio Cruz}

% Indicau el curs acadèmic que correspongui.
\date{Any acadèmic 2025-26}

% Indicau les paraules claus del vostre treball.
\paraulesclau{TFG, memòria, \LaTeX}

% Si l'autor no autoritza la publicació del treball
% descomentau la línia següent
%\autorfalse

% Si el tutor no autoritza la publicació del treball
% descomentau la línia següent
%\tutorfalse

%---------------------------------------------------------------------------------------------------------------------------------------------------------------

% Durant l'escriptura de la memòria haureu de 
% compilar-la moltes vegades. Si voleu guanyar
% una mica de temps, podeu dividir el contingut
% en diferents fitxers i compilar-ne sols alguns.
% La comanda següent és la que vos permet
% definir quins compilar i quins no.
%\includeonly{Instruccions,Annexos}

% Quan vulgueu treballar amb tot el document, 
% simplement, comentau la línia anterior.  

\begin{document}

% Recordau haver indicat, títol, autor i tutor
% i no toqueu les línies següents.
\portada
\portadainterior
\frontmatter

% Voleu dedicar i agrair el treball a algú?
% Activau les línies següents i escriu el
% que vulguis dins l'entorn 'agraiments' 
%
\cleartorecto \thispagestyle{empty}
\begin{agraiments}
Gràcies a tots els professors que m'han ajudat a arribar a aquesta fita dels meus estudis.
\end{agraiments}

% A continuació el Sumari
\cleartorecto \tableofcontents

% Si voleu que apareguin una llista de figures
% i taules, activau les línies corresponents
%\cleartorecto \listoffigures
%\cleartorecto \listoftables 

% Si apareixen molts acrònims a la documentació
% convindrà fer-ne una llista. Podeu veure com
% crear-la consultant el fitxer 'Acronims.tex',
% que és el que s'inclou aquí.
\chapter{Acrònims} %Respectau títol del capítol.
%
% Per utilitzar els acrònims es recomana fer un poc 
% de recerca bibliogràfica per entendre com 
% funcionen. Concretament podeu llegir el manual
% que teniu dins el vostre sistema.
% La comanda `texdoc acronym` hauria de mostrar-lo.
%
\begin{acronym}

\acro{EPS}[EPS]{Escola Politècnica Superior}

\acro{RDI}[R+D+I]{Recerca, Desenvolupament i Innovació}

\acro{TFG}[TFG]{Treball Final de Grau}


\end{acronym}
 
% Si no usau acrònims, comentau la línia anterior

% En l'arxiu Resum.tex es posarà el resum
% del treball.
%!TeX root=MemoriaTFG.tex

\chapter{Resum}

 

% No toqueu la línia següent 
\mainmatter\pagestyle{ruled}

%%%%%% COS DEL TREBALL %%%%%%%%%%%%

% Una bona pràctica consisteix en dividir
% un document llarg en diferents fragments,
% per exemple per capítols, i incloure aquests
% fragments dins el fitxer principal amb la
% comanda \include. Així, podem escriure la 
% comanda \includeonly{llista fitxers a compilar}
% que ens permetrà processar sols aquella part
% del document que ens interessi.

%!TeX root=MemoriaTFG.tex

\chapter{Introducció}

%!TeX root=MemoriaTFG.tex

\chapter{Instruccions generals i itinerari del Treball Final de Grau }\label{instruccions}
% \begin{figure}
% \centering
% \includegraphics[width=\linewidth]{Itinerari_TFG}
% \caption{\label{fig:itinerari}Itinerari del Treball Final de Grau}
% \end{figure}

\section{Matrícula del Treball Final de Grau}

Com a norma general, tal com apareix reflectit en els plans d'estudis de gairebé totes les titulacions de grau, el \ac{TFG} s'hauria de realitzar en el quart curs. Tanmateix, tot i que aquestes s'agrupen en cursos i semestres, els estudis universitaris s'estructuren en assignatures. La normativa de l'\acf{EPS} marca que per tal de poder-se matricular al \ac{TFG} l'estudiant ha d'haver superat, o estar matriculat, de totes les assignatures del pla d'estudis, llevat de (com a màxim) 18 ECTS optatius.

\section{Selecció del tema de Treball Final de Grau}

La via estàndard per escollir el tema de \ac{TFG} és a través de l'eina dedicada a la gestió dels \ac{TFG}, accessible des de la web  de l'\ac{EPS}. En aquest cas, després de revisar les propostes realitzades pels diferents professors involucrats en la docència de l'\ac{EPS}, escollirem la que més ens interessi i, a través de la mateix eina web, farem la sol·licitud corresponent. Aquesta sol·licitud serà revisada pel professor responsable de la proposta i, si aquesta és acceptada, ens assignarà el tema de \ac{TFG} i s'iniciarà el procés de realització del TFG.

Existeixen altres possibilitats a l'hora d'escollir un tema de \ac{TFG}. Per exemple, podem realitzar el \ac{TFG} en una empresa del sector. En aquest cas és important que contactem amb un professor de l'\ac{EPS} que vulgui realitzar les tasques de tutorització i que pertanyi a un àrea de coneixement propera als continguts a tractar en el \ac{TFG}, d'aquesta manera ens assegurarem que la proposta de \ac{TFG} i el camp que aquesta cobreix compleixen amb els estàndards habituals a la nostra titulació. A més, caldrà designar un supervisor de l'empresa que faci el seguiment del treball i coordini amb el tutor acadèmic. En cas de dubte, el més convenient és contactar amb el cap d'estudis de la titulació corresponent que segur que ens orientarà i ens proporcionarà la informació necessària.

Val a dir que el \ac{TFG} també es pot realitzar en una universitat amb la que s'hagin establert convenis de convalidació (programes Sèneca, Erasmus, Averroes, \ldots). Tanmateix, en aquest cas haurem de seguir els procediments administratius establerts en aquesta altra universitat.

% \section{Preparació de la proposta i contracte docent}

% Un cop escollit el tema de \ac{TFG} començarem a treballar en la fase de documentació i, d'acord amb el nostre supervisor, prepararem la proposta de \ac{TFG}. Per a la redacció d'aquesta proposta convé seguir les indicacions descrites al capítol \ref{proposta} i és recomanable que aquesta fase de documentació i preparació de la proposta de \ac{TFG}, tal com es mostra a la Fig. \ref{fig:itinerari}, no s'allargui més enllà d'un mes. Sobre la proposta de \ac{TFG} és sobre la que estudiant i professor signaran el contracte docent. En aquest contracte s'establiran els compromisos del professor quant a seguiment i supervisió del projecte i els de l'estudiant quant a dedicació i termini de presentació. La proposta de \ac{TFG} i el contracte docent, signat tant per l'alumne com pel professor, seran lliurats als serveis administratius on l'\ac{EPS} signarà el compromís de disponibilitat de medis materials genèrics i de formalització d'un tribunal de \ac{TFG} adient.

\section{Desenvolupament del Treball Final de Grau}

Un cop aprovada la sol·licitud de \ac{TFG} per part del tutor, ens dedicarem a la realització del \ac{TFG} i a la redacció de la memòria sota la supervisió del tutor i amb els recursos disponibles que s'han posat a la nostra disposició segons la informació donada en la proposta de \ac{TFG}. Per a la redacció de la memòria convé seguir les indicacions descrites al capítol \ref{memoria}. 

\section{Dipòsit de la memòria}

Una vegada acabada la seva redacció, i amb el vist-i-plau del nostre supervisor, dipositarem la memòria del \ac{TFG} a secretaria seguint les indicacions de la normativa de \acsp{TFG} de l'\ac{EPS}. Cal tenir en compte que el tutor haurà de validar la documentació dipositada per l'estudiant, i que en cas de no estar conforme amb alguna part de la memòria o altre documentació lliurada, podrà retornar-la a l'estudiant per tal que aquesta sigui revisada i corregida. 

De manera excepcional, el tutor podrà autoritzar la tramitació del dipòsit tot deixant constància expressa de la seva disconformitat amb el treball presentat, i que la tramitació es fa sota la responsabilitat exclusiva de l’estudiant. 

Un cop acceptada la tramitació, aquesta serà revisada pel tribunal de \ac{TFG} que s'hagi assignat al nostre treball. Si el tribunal considera que la memòria no compleix amb els requisits mínims, podrà retornar-la a l'estudiant per tal que aquesta sigui revisada i corregida. 



\section{Preparació de la presentació}

Finalment només ens quedarà preparar la presentació del \ac{TFG} per tal de fer-ne la defensa oral davant el tribunal. Per a la preparació d'aquesta presentació convé seguir les indicacions del capítol \ref{presentació}.

%\include{Proposta}
% !TEX root=MemoriaTFG.tex

\chapter{La memòria del treball de final de grau}\label{memoria}

% !TEX root=MemoriaTFG.tex

\chapter{La presentació del Treball Final de Grau}\label{presentació}


%!TeX root=MemoriaTFG.tex

\chapter{Conclusions}



%%%%%%% Fi cos del treball %%%%%%%%%%%

% Si el vostre document no conté apèndixs 
% comentau les dues línies següents
\appendix 
%!TeX root=MemoriaTFG.tex

\chapter{Format de la memòria}

 

% En aquest cas sols hi ha un fitxer d'annexos,
% però podeu afegir tants \include com calgui. 

%%%%%%% Fi apèndix 

% No toqueu la línia següent 
\backmatter

% La comanda següent defineix l'estil bibliogràfic
\bibliographystyle{IEEEtran}

% La comanda següent defineix el fitxer que
% conté les referències bibliogràfiques.
% En aquest cas és el fitxer Bibliografia.bib
\bibliography{Bibliografia} 

\end{document}